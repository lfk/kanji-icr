% Document title
% ==============
% Draft for conference/workshop paper to be submitted to http://tappcs.blogspot.mx/
% 2--4 pages in Spanish or English.

\documentclass[10pt,conference,a4paper]{IEEEtran}
\usepackage{hyperref}
%\usepackage{


\title{Some Nice Title Goes Here, Eventually\ldots}

\author{
	\IEEEauthorblockN{Lars Fredrik Karlstr\"om}
	\IEEEauthorblockA{Faculty of Science, Dept. of Computer Science\\ Universidad Aut\'onoma de Baja California\\ \texttt{fredrik.karlstrm@uabc.edu.mx}}
	\and
	\IEEEauthorblockN{Dr. Everardo Guti\'errez L\'opez}
	\IEEEauthorblockA{Faculty of Science, Dept. of Computer Science\\ Universidad Aut\'onoma de Baja California\\ \texttt{your-cool-academic-email-here@uabc.edu.mx}}
}

%\markboth{TACS 2014 and more info here}{} % Only on journal papers, apparently...

\begin{document}
	\maketitle

	\begin{abstract}
		Oh golly, this is where the abstract should go.
		``I can hardly wait to begin writing it!''
		
		Or well, let's see how things turn out in the body first, eh.
	\end{abstract}
	\medskip
	\begin{IEEEkeywords}
		Intelligent character recognition, Japanese writing, \ldots
	\end{IEEEkeywords}


	\section{Problem description}

	Any form of digital handwriting recognition is a complex undertaking,
	where the performance of the system is impacted by a multitude of factors -- spanning
	everything from the quality and quantity of training data, to resilience toward input
	translation, rotation, and noise. As the number of candidate categories increases
	the size of the search space, however, the task quickly grows monumental.

	The Japanese writing system consists of the two \emph{kana} syllabaries \emph{hiragana} and \emph{katakana},
	as well as the sinographs commonly known as \emph{kanji} -- literally ``Han characters'' -- that stem from China.
	The Japanese Ministry of Education's official ``regular use'' (j\=oy\=o) kanji list [ref] defines a set of 2,136
	baseline characters taught in elementary-- and secondary--school education. This set is far from exhaustive though:
	the Japanese Industrial Standard X 0208 encoding, for instance, contains 6,355 different kanji characters.

	Given that the category search space for Japanese characters is an order of magnitude larger than its western counterparts,
	it becomes apparent that a monolithic system design would be both unwieldly and computationally expensive, depending on
	deep neural networks [ref?] for high performance.

	In this work, we describe a flexble, modular design scheme that splits the identification task into coarse and fine steps,
	thereby reducing the complexity of each component. The design scheme also permits replacing classification strategies,
	and facilitates the parallelization of network training. 

	 


	\section{Previous work}

	Thus concludes this tale of Foo Bar, whose epic journey had a tremendous
	impact on the field of cryptography. Thank you, Foo, for your contributions.


	\section{Proposed solution}

	Some text here, followed by a diagram. Divide and conquer!

	%\begin{figure}
	%\centering
	%\includegraphics[width=2.5in]{myfigure}
	%\caption{Simulation Results}
	%\label{fig_sim}
	%\end{figure}

	\subsection{Stroke feature extraction}

	\subsection{Category clustering}

	\subsection{Feedforward routing network}

	\subsection{Convolutional character classification network}


	\section{Experiments}

	This will be a short section\ldots
	Describe the simple online feature extractor and following clustering scheme.


	\section{Conclusions}

	This work is cool, now give complimentary M.Sc. plx?
	They did it for Donald Knuth\ldots :(


	\section{References}

\end{document}
